
\definecolor{cobalt}{rgb}{0.0, 0.28, 0.67}

% matplotlibの色
\definecolor{C0}{rgb}{0.12156862745098039, 0.4666666666666667, 0.7058823529411765}
\definecolor{C1}{rgb}{1.0, 0.4980392156862745, 0.054901960784313725}
\definecolor{C2}{rgb}{0.17254901960784313, 0.6274509803921569, 0.17254901960784313}
\definecolor{C3}{rgb}{0.8392156862745098, 0.15294117647058825, 0.1568627450980392}
\definecolor{C4}{rgb}{0.5803921568627451, 0.403921568627451, 0.7411764705882353}
\definecolor{C5}{rgb}{0.5490196078431373, 0.33725490196078434, 0.29411764705882354}
\definecolor{C6}{rgb}{0.8901960784313725, 0.4666666666666667, 0.7607843137254902}
\definecolor{C7}{rgb}{0.4980392156862745, 0.4980392156862745, 0.4980392156862745}
\definecolor{C8}{rgb}{0.7372549019607844, 0.7411764705882353, 0.13333333333333333}
\definecolor{C9}{rgb}{0.09019607843137255, 0.7450980392156863, 0.8117647058823529}

% % 青, 赤, 黄色
\colorlet{CBlue}{cobalt}
\colorlet{CRed}{C3}
\colorlet{CYellow}{C1}

% 第1色,第2色, 第3色の定義
\colorlet{CA}{cobalt}
\colorlet{CB}{C3}
\colorlet{CC}{C1}
\colorlet{CText}{black!85}

% Beamerのフォントの色を変更する
\setbeamercolor{normal text}{fg=CText}
\setbeamercolor{titlelike}{fg=CA,bg=white}
\setbeamercolor{frametitle}{fg=CA,bg=white}

\setbeamercolor{description item}{fg=CText}


\newcommand{\TextCA}[1]{\textcolor{CA}{#1}}
\newcommand{\TextCB}[1]{\textcolor{CB}{#1}}
\newcommand{\TextCC}[1]{\textcolor{CC}{#1}}


% 文字に背景色を付けます
\newcommand{\Background}[2]{\tikz[baseline=(x.base)]{\node[rectangle,rounded corners,fill=#1!30](x){#2};}}
\newcommand{\Backgroundcap}[4][0cm]{\tikz[baseline=(x.base)]{\node[rectangle,rounded corners,fill=#2!30](x){#3} node[below=#1 of x, color=#2]{#4};}}

\newcommand{\BackgroundA}[1]{\Background{CA}{#1}}
\newcommand{\BackgroudAcap}[3][0cm]{\Background[#1]{CA}{#2}{#3}}
\newcommand{\BackgroundB}[1]{\Background{CB}{#1}}
\newcommand{\BackgroudBcap}[3][0cm]{\Background[#1]{CB}{#2}{#3}}
\newcommand{\BackgroundC}[1]{\Background{CC}{#1}}
\newcommand{\BackgroudCcap}[3][0cm]{\Background[#1]{CC}{#2}{#3}}


% 文字を線で囲みます
\newcommand{\RoundedRect}[2]{\tikz[baseline=(x.base)]{\node[rectangle,rounded corners,draw=#1,line width=0.5mm](x){#2};}}
\newcommand{\RoundedRectcap}[4][0cm]{\tikz[baseline=(x.base)]{\node[rectangle,rounded corners,draw=#2,line width=0.5mm](x){#3} node[below=#1 of x, color=#2]{#4};}}
\newcommand{\RoundedRectDash}[3]{\tikz[baseline=(x.base)]{\node[rectangle,rounded corners,draw=#1,line width=0.5mm,#2](x){#3};}}
\newcommand{\RoundedRectDashcap}[5][0cm]{\tikz[baseline=(x.base)]{\node[rectangle,rounded corners,draw=#2,line width=0.5mm,#3](x){#4} node[below=#1 of x, color=#2]{#5};}}

\newcommand{\RoundedRectA}[1]{\RoundedRectDash{CA}{dashed}{#1}}
\newcommand{\RoundedRectAcap}[3][0cm]{\RoundedRectDash[#1]{CA}{dashed}{#2}{#3}}
\newcommand{\RoundedRectB}[1]{\RoundedRectDash{CB}{dotted}{#1}}
\newcommand{\RoundedRectBcap}[3][0cm]{\RoundedRectDash[#1]{CB}{dotted}{#2}{#3}}
\newcommand{\RoundedRectC}[1]{\RoundedRectDash{CC}{solid}{#1}}
\newcommand{\RoundedRectCcap}[3][0cm]{\RoundedRectDash[#1]{CC}{solid}{#2}{#3}}


\newcommand{\UnderlineDashcap}[6][0cm]{%
  \tikz[baseline=(x.base)]{%
    \node[inner sep=1.5pt,outer ysep=0pt,outer xsep=-2pt,color=#2] (x) {#5};
    \ifthenelse{\equal{#4}{zigzag}}{%
      \draw[decorate,decoration={zigzag,amplitude=0.4mm,segment length=2mm},color=#3,line width=0.5mm] (x.south west) -- (x.south east);
    }{\ifthenelse{\equal{#4}{snake}}{%
      \draw[decorate,decoration={snake,amplitude=0.4mm,segment length=2mm},color=#3,line width=0.5mm] (x.south west) -- (x.south east);
    }{\ifthenelse{\equal{#4}{coil}}{%
      \draw[decorate,decoration={coil,amplitude=0.4mm,segment length=2mm},color=#3,line width=0.5mm] (x.south west) -- (x.south east);
    }{\ifthenelse{\equal{#4}{bumps}}{%
      \draw[decorate,decoration={bumps,amplitude=0.8mm,segment length=4mm},color=#3,line width=0.5mm] (x.south west) -- (x.south east);
    }{\draw[#4,line width=0.5mm,color=#3,line width=0.5mm] (x.south west) -- (x.south east);
    }}}}%
    \ifthenelse{\equal{#6}{}}{}{%
      \node[below=#1 of x, color=#3]{#6};
    }%
  }%
}%

\newcommand{\UnderlineDash}[4]{\UnderlineDashcap[0cm]{#1}{#2}{#3}{#4}{}}


\newcommand{\UnderlineA}[1]{\UnderlineDash{CText}{CA}{solid}{#1}}
\newcommand{\UnderlineAcap}[3][0cm]{\UnderlineDashcap[#1]{CText}{CA}{solid}{#2}{#3}}
\newcommand{\UnderlineB}[1]{\UnderlineDash{CText}{CB}{dashed}{#1}}
\newcommand{\UnderlineBcap}[3][0cm]{\UnderlineDashcap[#1]{CText}{CB}{dashed}{#2}{#3}}
\newcommand{\UnderlineC}[1]{\UnderlineDash{CText}{CC}{zigzag}{#1}}
\newcommand{\UnderlineCcap}[3][0cm]{\UnderlineDashcap[#1]{CText}{CC}{zigzag}{#2}{#3}}

\newcommand{\EmpA}[1]{\Background{CA}{#1}}
\newcommand{\EmpAcap}[3][0cm]{\Background[#1]{CA}{#2}{#3}}
\newcommand{\EmpB}[1]{\Background{CB}{#1}}
\newcommand{\EmpBcap}[3][0cm]{\Background[#1]{CB}{#2}{#3}}
\newcommand{\EmpC}[1]{\Background{CC}{#1}}
\newcommand{\EmpCcap}[3][0cm]{\Background[#1]{CC}{#2}{#3}}

% \btVFillを入れることで,ページの下部にテキストを表示できる
\newcommand{\btVFill}{\vskip0pt plus 1filll}
% \btVFillを活用したFootnote環境
\newenvironment{Footnote}[1][8pt]{
  \btVFill
  % \begin{spacing}{0.5}
  \fontsize{#1}{0pt}\selectfont
}{}


% カラーボックス tcolorbox の設定
% These options will be applied to all `tcolorboxes`
\tcbuselibrary{raster}
\tcbset{%
  noparskip,
  colback=gray!10, %background color of the box
  colframe=gray!40, %color of frame and title background
  coltext=black, %color of body text
  coltitle=black, %color of title text 
  fonttitle=\gtfamily,
  left=4pt, right=4pt,
}

% alertbox と infobox の宣言
\newtcolorbox{alertbox}[1]{colback=white, colframe=CB!90, fonttitle=\gtfamily, coltext=black!80, coltitle=white, title=#1}
\newtcolorbox{infobox}[1]{colback=white, colframe=CA!90, fonttitle=\gtfamily, coltext=black!80, coltitle=white, title=#1}

\newtcolorbox{rightnote}{left=0pt, right=0pt, top=0pt, bottom=0pt, colback=CA!15, colframe=white, width=\dimexpr\textwidth\relax, enlarge left by=0mm, boxsep=5pt, arc=0pt, outer arc=0pt}

% \newcommand{\headuline}[2]{\textcolor{#1}{#2} \bgroup\markoverwith{\textcolor{#1!50}{\rule[0.5ex]{2pt}{1.5pt}}}\ULon{\hfill}}

\newcommand{\headuline}[2]{\textcolor{#1}{#2}\textcolor{#1!40}{\dotfill}}

\newcommand{\rightnotebox}[2]{
\begin{rightnote}
\footnotesize
\ifx&#1&%
% #1 is empty
#2
\else
% #1 is nonempty
\textcolor{CA!90}{#1}
#2
\fi
\end{rightnote}
}


% 使い方.0.5の部分を省略するとbodyとnoteが0.77:0.21になる
% \framewithnote[0.5]{テスト3}
% {
% body
% }
% {
% note
% }
\newcommand{\framewithnote}[4][.76]{
\ifisWideFrame
  % 16:9
  \renewcommand{\frametitleline}{\par\vskip-12pt\rule{#1\textwidth}{0.1pt}}
  \begin{frame}[t]{#2}
  \begin{columns}[T,onlytextwidth]
  \begin{column}{#1\textwidth}
  % left column
  #3
  \end{column}
  \begin{column}{.01\textwidth}
  \end{column}
  \begin{column}{\textwidth-#1\textwidth-.01\textwidth+0.75cm}
  \vspace{-\baselineskip}% 右カラムは上に詰める
  \vspace{-0.2cm}% 右カラムは上に詰める
  \ifx&#4&%
  % #4 is empty
  \rightnotebox{Note...}{
  \vspace{\textheight-1.75cm}
  }
  \else
  % #4 is nonempty
  #4
  \fi
  \end{column}
  \end{columns}
  \end{frame}
  \renewcommand{\frametitleline}{\par\vskip-12pt\hrulefill}% タイトル下の線をもとに戻す
\else
% 4:3
  \begin{frame}[t]{#2}
    #3
  \end{frame}
\fi
}

% 使い方.0.2の部分を省略するとbodyと余白の比が(1-0.11*2):(0.11*2)になる
% \centerframe[0.2]{テスト3}
% {
% body
% }
\newcommand{\centerframe}[3][.11]{
\begin{frame}[t]{\hspace{#1\textwidth}#2}
% \vspace{-\baselineskip}
\begin{columns}[T,onlytextwidth]
\begin{column}{#1\textwidth}
\end{column}
\begin{column}{\textwidth-#1\textwidth-#1\textwidth}
#3
\end{column}
\begin{column}{#1\textwidth}
\end{column}
\end{columns}
\end{frame}
}