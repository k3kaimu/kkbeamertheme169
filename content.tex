% !TEX root = template.tex

% \AtBeginSection[]{
%     \begin{frame}{目次}
%         \tableofcontents[currentsection, hideallsubsections]
%     \end{frame}
% }

\def\SlideTitle{スライドのタイトルです}

\author[私です]{小松和暉}
\title[スライドの例です]{\SlideTitle}
\institute[所属です]{豊橋技術科学大学}
\date{\today}

\begin{frame}[t,plain,noframenumbering]{}
    % ロゴを入れる
    \begin{tikzpicture}[overlay]
        \node at (\paperwidth-12ex,-1.2cm) {
          \includegraphics[width=7ex]{logo/tut_logo.pdf}
        };
    \end{tikzpicture}%

    \vspace{2cm}
    % \vfill

    \begin{rightnote}
    \begin{center}
        {\LARGE
        XXXXXに関する調査
        }
    \end{center}
    \end{rightnote}

    % \vspace{0.5cm}
    \vfill

    \begin{center}
        \begin{tabular}{rrl}
            $\dagger$〇〇〇大学&〇&山田太郎 \\
            $\dagger\dagger$〇〇〇大学&&天伯太郎 \\
            $\dagger$&&雲雀四郎之助 \\
        \end{tabular}

        % \vspace{0.5cm}
        \vfill

        {
        AAAAAAAA学会 AAAAAAAA研究会\\
        X月研究会@YYYYY大学
        }
    \end{center}
\end{frame}


\framewithnote{フーリエ変換とその逆変換}
{
\headuline{CA}{\textbf{フーリエ変換}}

\UnderlineB{信号$x(t)$}から\UnderlineA{周波数成分$X(f)$}を求める積分変換
\begin{align*}
\UnderlineA{$X(f)$} = \int_{-\infty}^{\infty} \UnderlineB{$x(t)$} \mathrm{e}^{-j2\pi ft} \mathrm{d}t
\end{align*}

\vspace{1em}
\headuline{CA}{\textbf{逆フーリエ変換}}

\UnderlineA{周波数成分$X(f)$}から\UnderlineB{信号$x(t)$}を再構成する積分変換
\begin{align*}
    \UnderlineB{$x(t)$} = \int_{-\infty}^{\infty} \UnderlineA{$X(f)$} \mathrm{e}^{j2\pi ft} \mathrm{d}f
\end{align*}

}
{
\rightnotebox{メモ1}{\\
\UnderlineA{周波数成分$X(f)$}を信号の\textbf{スペクトル}と呼ぶ
}
\rightnotebox{注意1}{\\
順変換と逆変換で$\mathrm{e}$の肩に乗っている$j2\pi f t$の符号が異なることに注意
}
}


\framewithnote{その2}{
\begin{infobox}{あいうえお}
\EmpA{かき}くけこ
\end{infobox}
\begin{alertbox}{さしすせそ}
たちつてと
\end{alertbox}

% \headuline{\textcolor{CA}{みだし}\hfill}
\headuline{CA}{\textbf{逆フーリエ変換}}


\headuline{CB}{みだし}
}{}

\begin{frame}[t]{タイトル}
    \begin{itemize}
        \item<1-> 項目1
        \item<2-> 項目2
        \item<3-> 項目3
    \end{itemize}
\end{frame}
